\documentclass[12pt]{article}
\usepackage[utf8]{inputenc}
\usepackage[
top=2cm,
bottom=2cm,
left=3cm,
right=2cm,
headheight=17pt, % as per the warning by fancyhdr
includehead,includefoot,
heightrounded, % to avoid spurious underfull messages
]{geometry} 
\geometry{a4paper}
\usepackage[ngerman]{babel}
\usepackage{listings}
\usepackage{fancyhdr}
\usepackage{siunitx}
\usepackage{float}
\usepackage{graphicx}
\usepackage{caption}
\usepackage[table]{xcolor}
\usepackage{diagbox}
\usepackage{lipsum}

% Assembler
\lstdefinelanguage
{Assembler} % based on the "x86masm" dialect
% with these extra keywords:
{morekeywords={call, mov}} % etc.

% Lecture Name, exercise number, group number/members
\newcommand{\lecture}{Parallel Computer Architecture}
\newcommand{\exercise}{Exercise 7}
\newcommand{\groupnumber}{Group 04}
\newcommand{\groupmembersshort}{Barley, Barth, Nisblé}
\newcommand{\groupmemberslist}{Barley, Daniel\\Barth, Alexander\\Nisblé, Patrick}
\newcommand{\duedate}{2020-01-15, 12:00}

\fancyhf{}
\fancyhead[L]{\groupnumber}
\fancyhead[R]{\textsc{\groupmembersshort}}
\fancyfoot[C]{\lecture: \exercise}
\fancyfoot[R] {\thepage}
\renewcommand{\headrulewidth}{0.4pt}
\renewcommand{\footrulewidth}{0.4pt}
\pagestyle{fancy}

\begin{document}
	\begin{titlepage}
		\centering
		
		{\scshape\LARGE Universität Heidelberg\\Institute for Computer Engineering (ZITI) \par}
		\vspace{1.5cm}
		{\scshape\Large Master of Science Computer Engineering \par}
		\vspace{0.5cm}
		{\scshape\Large \lecture \par}
		\vspace{1.5cm}
		{\huge\bfseries \exercise \par}
		\vspace{2cm}
		{\Large \groupnumber \itshape  \\ \groupmemberslist \par}
		\vfill
		
		
		% Bottom of the page
		{\large Due date \duedate \par}
	\end{titlepage}

\setcounter{section}{7}
%\subsection{Ping-Pong und Ping-Exchange in MPI}

%@todo: code

%\subsection{Experimente und Evaluation}

%@todo: complete

\setcounter{subsection}{2}
\subsection{Hardware Affinität und Performance}

\noindent \textbf{a.}

Wird ein Prozess an bestimmte NUMA Nodes oder CPU Cores gebunden, so erfolgt der Speicherzugriff lokal.
Dies ist schneller als auf einen entfernten Speicher, da keine Verzögerung durch das Verbindungsnetzwerk erfolgt.
Das Verbindungsnetzwerk führt den globalen Zugriff aus.
Durch die Reduzierung der globalen Kommunikationslast durch das Binden der Prozesse an NUMA Nodes oder CPU Cores, verringern sich die Anforderungen an das Verbindungsnetzwerk. 

%\noindent \textbf{b.}

%@todo: complete

\end{document}